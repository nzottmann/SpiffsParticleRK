{\itshape Port of the S\+P\+I\+F\+FS library for the Particle platform}

\subsection*{Introduction}

The \href{https://github.com/pellepl/spiffs}{\tt excellent S\+P\+I\+F\+FS library} provides a simple file system on a N\+OR flash chip. This is a port of the library for the Particle platform, with a few convenience helpers to make using it easier from C++ and using standard Particle/\+Arduino/\+Wiring A\+P\+Is.

Both the original S\+P\+I\+F\+FS library and this port and M\+IT licensed, so you can use them in open-\/source or in closed-\/source commercial products without a fee or license.

\subsection*{Flash Support}

This library supports S\+P\+I-\/connected N\+OR flash chips. These are typically tiny 8-\/\+S\+O\+IC surface mount chips, intended to be included on your own circuit board. There are also breadboard adapters that are available, shown in the examples below.

The chips are really inexpensive, less than US\$0.\+60 in single quantities for a 1 Mbyte flash. They\textquotesingle{}re available in sizes up to 16 Mbyte.

S\+PI flash is less expensive than SD cards, and do not need an adapter or card slot. Of course they\textquotesingle{}re not removable, either.

The underlying \href{https://github.com/rickkas7/SpiFlashRK}{\tt Spi\+Flash\+RK library} library supports S\+PI N\+OR flash from


\begin{DoxyItemize}
\item I\+S\+SI, such as a \href{http://www.digikey.com/product-detail/en/issi-integrated-silicon-solution-inc/IS25LQ080B-JNLE/706-1331-ND/5189766}{\tt I\+S25\+L\+Q080B}.
\item Winbond, such as a \href{https://www.digikey.com/product-detail/en/winbond-electronics/W25Q32JVSSIQ/W25Q32JVSSIQ-ND/5803981}{\tt W25\+Q32}.
\item Macronix, such as the \href{https://www.digikey.com/product-detail/en/macronix/MX25L8006EM1I-12G/1092-1117-ND/2744800}{\tt M\+X25\+L8006\+E\+M1\+I-\/12G}
\item The external 1 Mbyte flash on the P1 module
\item Probably others
\end{DoxyItemize}

It does not support I2C flash, SD cards, or non-\/flash chips like F\+R\+AM.

\subsection*{Instantiating an Spi\+Flash object}

You typically instantiate an object to interface to the flash chip as a global variable\+:


\begin{DoxyCode}
SpiFlashISSI spiFlash(SPI, A2);
\end{DoxyCode}


I\+S\+SI flash connected to the primary S\+PI with A2 as the CS (chip select or SS).


\begin{DoxyCode}
SpiFlashWinbond spiFlash(SPI, A2);
\end{DoxyCode}


Winbond flash connected to the primary S\+PI with A2 as the CS (chip select or SS).


\begin{DoxyCode}
SpiFlashWinbond spiFlash(SPI1, D5);
\end{DoxyCode}


Winbond flash connected to the secondary S\+PI, S\+P\+I1, with D5 as the CS (chip select or SS).


\begin{DoxyCode}
SpiFlashMacronix spiFlash(SPI1, D5);
\end{DoxyCode}


Macronix flash connected to the secondary S\+PI, S\+P\+I1, with D5 as the CS (chip select or SS). This is the recommended for use on the E-\/\+Series module. Note that this is the 0.\+154", 3.\+90mm width 8-\/\+S\+O\+IC package.


\begin{DoxyCode}
SpiFlashP1 spiFlash;
\end{DoxyCode}


The external flash on the P1 module. This extra flash chip is entirely available for your user; it is not used by the system firmware. You can only use this on the P1; it relies on system functions that are not available on other devices.

\subsection*{Connecting the hardware}

The real intention is to reflow the 8-\/\+S\+O\+IC module directly to your custom circuit board. However, for prototyping, here are some examples\+:

For the primary S\+PI (S\+PI)\+:

\tabulinesep=1mm
\begin{longtabu} spread 0pt [c]{*{4}{|X[-1]}|}
\hline
\rowcolor{\tableheadbgcolor}\textbf{ Name  }&\textbf{ Flash Alt Name  }&\textbf{ Particle Pin  }&\textbf{ Example Color   }\\\cline{1-4}
\endfirsthead
\hline
\endfoot
\hline
\rowcolor{\tableheadbgcolor}\textbf{ Name  }&\textbf{ Flash Alt Name  }&\textbf{ Particle Pin  }&\textbf{ Example Color   }\\\cline{1-4}
\endhead
SS  &CS  &A2  &White   \\\cline{1-4}
S\+CK  &C\+LK  &A3  &Orange   \\\cline{1-4}
M\+I\+SO  &DO  &A4  &Blue   \\\cline{1-4}
M\+O\+SI  &D1  &A5  &Green   \\\cline{1-4}
\end{longtabu}


For the secondary S\+PI (S\+P\+I1)\+:

\tabulinesep=1mm
\begin{longtabu} spread 0pt [c]{*{4}{|X[-1]}|}
\hline
\rowcolor{\tableheadbgcolor}\textbf{ Name  }&\textbf{ Flash Alt Name  }&\textbf{ Particle Pin  }&\textbf{ Example Color   }\\\cline{1-4}
\endfirsthead
\hline
\endfoot
\hline
\rowcolor{\tableheadbgcolor}\textbf{ Name  }&\textbf{ Flash Alt Name  }&\textbf{ Particle Pin  }&\textbf{ Example Color   }\\\cline{1-4}
\endhead
SS  &CS  &D5  &White   \\\cline{1-4}
S\+CK  &C\+LK  &D4  &Orange   \\\cline{1-4}
M\+I\+SO  &DO  &D3  &Blue   \\\cline{1-4}
M\+O\+SI  &D1  &D2  &Green   \\\cline{1-4}
\end{longtabu}


Note that the S\+S/\+CS line can be any available G\+P\+IO pin, not just the one specified in the table above.


\begin{DoxyItemize}
\item Electron using Primary S\+PI
\end{DoxyItemize}


\begin{DoxyItemize}
\item Photon using Secondary S\+PI (S\+P\+I1)
\end{DoxyItemize}


\begin{DoxyItemize}
\item Photon using Primary S\+PI and a poorly hand-\/soldered 8-\/\+S\+O\+IC adapter
\end{DoxyItemize}


\begin{DoxyItemize}
\item E-\/series evaluation kit with a \href{https://www.digikey.com/product-detail/en/macronix/MX25L8006EM1I-12G/1092-1117-ND/2744800}{\tt Macronix M\+X25\+L8006\+E\+M1\+I-\/12G} soldered to the E-\/series module (outlined in pink).
\end{DoxyItemize}


\begin{DoxyItemize}
\item P1 module (extra flash is under the can)
\end{DoxyItemize}

 \subsection*{Setting up the \mbox{\hyperlink{class_spiffs_particle}{Spiffs\+Particle}} object}

\subsection*{Using the S\+P\+I\+F\+FS A\+PI}

\subsubsection*{Background}

A few things about S\+P\+I\+F\+FS\+:


\begin{DoxyItemize}
\item It\textquotesingle{}s relatively small, way smaller than S\+D\+F\+AT at least.
\item It has the bare minimum of things you need to store files.
\item You can allocate all or part of the flash chip to S\+P\+I\+F\+FS.
\end{DoxyItemize}

In particular, there are two important limitations\+:


\begin{DoxyItemize}
\item It does not support subdirectories; all files are in a single directory.
\item Filenames are limited to 31 characters.
\item It does not have file timestamps (modification or creation times).
\end{DoxyItemize}

Within the scope of how you use S\+P\+I\+F\+FS these shouldn\textquotesingle{}t be unreasonable limitations, though you should be aware.

Like most file systems, there are a few things you must do\+:


\begin{DoxyItemize}
\item You must mount the file system, typically at startup.
\item If your flash is blank, you\textquotesingle{}ll need to format the file system.
\item You can iterate the top level directory to find the file names in it.
\item In order to use data in a file, you must open it and get a file handle, read or write, then close it.
\item There are a finite number of files that can be open, but you can set the maximum. The default is 4.
\item If you think the file system is corrupted, you can check and repair it.
\end{DoxyItemize}

Once you get it working, there is some fine-\/tuning you can do\+:


\begin{DoxyItemize}
\item The cache size is programmable, which speeds up operations at the cost of additional R\+AM. The default is 2 logical blocks.
\item The logical block size is programmable. Using larger blocks can make using large files more efficient in flash storage, at the cost of more R\+AM and making small files less efficient in flash storage. The default is 4K, and can\textquotesingle{}t be made smaller but can be made larger.
\end{DoxyItemize}

\subsection*{Using the Arduino/\+Wiring File A\+PI}

\subsection*{Resource usage}